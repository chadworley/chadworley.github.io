\documentclass[12pt]{article}
\usepackage[T1]{fontenc}
\usepackage[utf8]{inputenc}
\usepackage[hidelinks]{hyperref}
\usepackage{fancyhdr}
\usepackage{tasks}
\settasks{label-format=\bfseries, label-width=2em, before-skip =\smallskipamount, after-item-skip=100pt}
\usepackage{wrapfig}
\usepackage[shortlabels]{enumitem}
\usepackage{amsmath}
\usepackage{xcolor}
\usepackage{newtxmath}
\usepackage{calligra}
\usepackage{titling}
\usepackage{lastpage}
\title{Syllabus for Precalculus 2023-2024}
\usepackage[total={7in, 9in},bottom=1in]{geometry}
\setlength{\headheight}{18pt}
%\setlength{\parindent}{0pt}
\usepackage{sagetex}

\fancypagestyle{firststyle}
{
   \fancyhead[L]{\calligra{\thetitle}}
  \fancyhead[R]{\bf \large Name:\hspace{2in}}
  \fancyfoot[R]{\sc Page \thepage \ of \pageref*{LastPage}}
  \fancyfoot[C]{}
}

\fancyhead[L]{}
\fancyhead[R]{}
\fancyhead[C]{\href{chadworley.github.io/precalculus
/syllabus/syllabus_precalc_23-24.pdf}{chadworley.github.io/precalculus
/syllabus/syllabus\_precalc\_23-24.pdf}}
\fancyfoot[R]{\sc Page \thepage \ of \pageref*{LastPage}}
\fancyfoot[C]{}
\pagestyle{fancy}

\usepackage[yyyymmdd]{datetime}
\renewcommand{\dateseparator}{--}

\begin{document}

\section*{Precalculus Syllabus 2023-2024}

Rendered \today.

\subsection*{Teacher}

\begin{itemize}
\item Chad Worley
\item chad.worley@bartcharter.org
\end{itemize}

\subsection*{Course Description}

\begin{itemize}
\item Statistics
\item Functions
\item Exponential and Logarithmic Functions
\item Trigonometry
\item Vectors, Complex Numbers, and Polar Functions
\item Conic Sections
\end{itemize}

Precalculus readies students for college-level math courses by previewing advanced topics while reviewing key algebraic, graphical, and computational skills. As the name "precalculus" implies, some key concepts and skills from calculus are addressed. However, in our class, we also dive into Statistics, which is not foundational to calculus.

Statistics is a branch of math that helps us make predictions about populations based on smaller samples. It helps us answer questions like whether dice are unfairly weighted or whether a new medicine is beneficial. These questions are difficult to answer because of natural variability in data: not every patient will die without the treatment, and not every patient will survive with the treatment. To understand this variability, we use probability, which is a way of modeling and understanding uncertain events.

After Statistics, we review function features. We use limits to describe discontinuities and end behavior (as $x\to\pm\infty$). We apply the average slope over a small interval to hint at derivatives. Inverse functions and composition (nested functions) are also discussed.

We then learn about exponential and logarithmic functions. These are important examples of inverses: using logarithms allows us to solve exponential equations and vice-versa. We see  exponential functions are created when growth (or decay) rate is proportional to the current amount. We explore this by investigating arithmetic, quadratic, and geometric sequences. A quick introduction to Euler's method introduces a powerful numerical method for integrating systems of differential equations, allowing some exploration of chaotic systems. Students use the geometric series formula to calculate the perimeter and area of fractals, which are intimately linked to chaos.

In Trigonometry we review right-triangle relationships between angle and side ratios. We extend the definition of these trigonometric ratios by using the unit circle. We see these periodic functions are useful for modeling waves. We will learn a bit about Fourier Transforms. Students will produce a hybrid image as a project by combining a low-frequency band-pass from one image and a high-frequency band-pass from another image.

Using the trigonometry skills, we will dive into vectors and complex numbers. Vectors are objects with both magnitude and direction, which we can represent in the Cartesian plane. Complex numbers are numbers that can have real and imaginary parts and have their own unique arithmetic rules. We will also explore polar form, which allows us to do quick multiplication with complex numbers. We will use technology to investigate polar and parametric functions.

Lastly, we will cover conic sections, which include parabolas, circles, ellipses, and hyperbolas. We will learn about their different properties and how their equations determine their features such as focal points and vertices.

\subsection*{Essential Questions}

\begin{itemize}
\item How can mathematical models and simulations help us understand and predict complex systems?
\item How do you prove a coin is fair?
\item What is the interval of typical totals when rolling 100 6-sided dice?
\item How do we undo functions with inverse functions?
\item How do you determine speed of a bicycle at a specific moment?
\item What are the implications of a rate of growth  proportional to the independent variable?
\item What are the implications of a rate of growth  proportional to the dependent variable?
\item Can we calculate the perimeter of a fractal?
\item Why do we discuss right triangles, circles, and waves in trigonometry; how are they related?
\item Where do parabolas, ellipses, and hyperbolas show up in nature (trajectories) and engineering?
\item How can we effectively communicate mathematical ideas and conclusions using appropriate mathematical language and notation?
\end{itemize}

\subsection*{Learning Outcomes}

\begin{itemize}
\item S.ID.A: Summarize, represent, and interpret data on a single count or measurement variable. Use calculators, spreadsheets, and other technology as appropriate.
\item S.ID.B: Summarize, represent, and interpret data on two categorical and quantitative variables.
\item S.ID.C: Interpret linear models.
\item S.IC.A: Understand and evaluate random processes underlying statistical experiments. Use calculators, spreadsheets, and other technology as appropriate.
\item S.IC.B: Make inferences and justify conclusions from sample surveys, experiments, and observational studies.
\item S.CP.A: Understand independence and conditional probability and use them to interpret data from simulations or experiments.
\item S.CP.B: Use the rules of probability to compute probabilities of compound events in a uniform probability model.
\item S.MD.A: Calculate expected values and use them to solve problems.
\item S.MD.B: Use probability to evaluate outcomes of decisions.
\item F.IF.A: Understand the concept of a function and use function notation.
\item F.IF.B: Interpret functions that arise in applications in terms of the context (linear, quadratic, exponential, rational, polynomial, square root, cube root, trigonometric, logarithmic).
\item F.IF.C: Analyze functions using different representations.
\item F.BF.A: Build a function that models a relationship between two quantities.
\item F.BF.B: Build new functions from existing functions.
\item N.RN.A.1:  Explain how the definition of the meaning of rational exponents follows from extending the properties of integer exponents to those values, allowing for a notation for radicals in terms of rational exponents.
\item N.RN.A.2: Rewrite expressions involving radicals and rational exponents using the properties of exponents.
\item F.BF.A.2: Write arithmetic and geometric sequences both recursively and with an explicit formula, use them to model situations, and translate between the two forms.
\item F.LE.A: Construct and compare linear, quadratic, and exponential models and solve problems.
\item F.LE.B: Interpret expressions for functions in terms of the situation they model.
\item F.BF.B.5 Understand the inverse relationship between exponents and logarithms and use this relationship to solve problems involving logarithms and exponents.
\item G.SRT.C: Define trigonometric ratios and solve problems involving right triangles.
\item F.TF.A: Extend the domain of trigonometric functions using the unit circle.
\item F.TF.B: Model periodic phenomena with trigonometric functions. 
\item F.TF.C: C. Prove and apply trigonometric identities.
\item N.VM.A: Represent and model with vector quantities.
\item N.VM.B: Perform operations on vectors.
\item (+) N.VM.C: Perform operations on matrices and use matrices in applications.
\item N.CN.A: Perform arithmetic operations with complex numbers.
\item N.CN.B: Represent complex numbers and their operations on the complex plane.
\item N.CN.C: Use complex numbers in polynomial identities and equations.
\item G.GMD.B.4: Identify the shapes of two-dimensional cross sections of three-dimensional objects, and identify three dimensional objects generated by rotations of two-dimensional objects.
\item G.GPE.A: Translate between the geometric description and the equation for a conic section.
\item G.GPE.A.3: Derive the equations of ellipses and hyperbolas given the foci, using the fact that the sum ordifference of distances from the foci is constant.
\item G.GPE.A.3.a: Use equations and graphs of conic sections to model real-world problems.
* Assessments
\end{itemize}



\subsection*{Evaluation}

At BART, students are evaluated on two key components: effort and mastery. In this class, each component is worth half (50\%) of the total grade. The effort grade is dependent on showing engagement with frequent, small tasks like entrance tickets, exit tickets, and homework. The mastery grade is dependent on performance in the larger tasks: the unit exams and unit projects.

\begin{sagesilent}
g = Graphics()
g += line([(65,0),(100,0)],axes=False,xmin=65,xmax=100,ymin=-3,ymax=3,aspect_ratio=1)
nums = [70,73.33,76.66,80,83.33,86.66,90,93.33,96.66,100]
lets = ["C-","C","C+","B-","B","B+","A-","A","A+"]
for i in nums:
    g += line([(i,-1),(i,3)])
    g += text(round(i,2),(i,-2))
for i in range(len(nums)-1):
    pos = (nums[i]+nums[i+1])/2
    g += text(lets[i],(pos,1))
g += text("no credit",(67,1))
\end{sagesilent}

\begin{itemize}
\item Grading categories
\begin{itemize}
\item Effort (50\%)
\begin{itemize}
\item Entrance tickets
\item Exit tickets
\item Homework completion
\end{itemize}
\item Mastery (50\%)
\begin{itemize}
\item Exams
\item Projects
\end{itemize}
\end{itemize}
\item Grading Scale \newline \sageplot{g,fig_tight=True}
\end{itemize}


\subsection*{Texts and Materials}

You may find the following open-source textbooks helpful.

\begin{itemize}
\item \href{https://cocalc.com/BART/math/precalc/files/textbooks/openIntroStats.pdf}{Stats: https://cocalc.com/BART/math/precalc/files/textbooks/openIntroStats.pdf}
\item \href{https://math.libretexts.org/Bookshelves/Precalculus}{Precalc: https://math.libretexts.org/Bookshelves/Precalculus}
\end{itemize}

\subsection*{Classroom Policies}

BART is implementing a no-tolerance policy toward electronic devices (except the chromebooks). This includes phones, earbuds, and smart watches. If I see (or hear) any of these devices in your possession, I will ask you to deliver it to Mr. Marcus Haas at Student Support, and I will make a written report of the infraction.

As the school year progresses, we will implement other classroom policies as necessary to foster an effective learning environment.


\end{document}